\documentclass{article}

\usepackage{fullpage}
\usepackage{tabularx}
\usepackage{booktabs}
\usepackage{setspace}
\doublespacing

\newcommand{\pkg}[1]{\texttt{#1}}
\newcommand{\cmd}[1]{\texttt{#1}}
\newcommand{\prog}[1]{{\sf #1}}
\newcommand{\proglang}[1]{\prog{#1}}
\newcommand{\R}{\prog{R}}

\usepackage{longtable}
\usepackage{pdflscape}
\usepackage[T1]{fontenc}
\usepackage{listings}
\usepackage{color}
\definecolor{lightgray}{rgb}{.9,.9,.9}
\definecolor{darkgray}{rgb}{.4,.4,.4}
\definecolor{purple}{rgb}{0.65, 0.12, 0.82}

\lstdefinelanguage{JavaScript}{
  keywords={typeof, new, true, false, catch, function, return, null, catch, switch, var, if, in, while, do, else, case, break},
  keywordstyle=\color{blue}\bfseries,
  ndkeywords={class, export, boolean, throw, implements, import, this},
  ndkeywordstyle=\color{darkgray}\bfseries,
  identifierstyle=\color{black},
  sensitive=false,
  comment=[l]{//},
  morecomment=[s]{/*}{*/},
  commentstyle=\color{purple}\ttfamily,
  morestring=[b]',
  morestring=[b]"
}

\lstset{
   language=JavaScript,
   extendedchars=true,
   basicstyle=\footnotesize\ttfamily,
   showstringspaces=false,
   showspaces=false,
   numbers=left,
   numberstyle=\footnotesize,
   numbersep=9pt,
   tabsize=2,
   breaklines=true,
   showtabs=false,
   captionpos=b,
   upquote=true
}

\usepackage[
    backend=biber,
    style=apa,
    sorting=nty
]{biblatex}
\addbibresource{references.bib}

\usepackage{authblk}
\usepackage{graphicx}
\usepackage[acronym]{glossaries}

\makeglossaries
\newacronym{iife}{IIFE}{instantaneously invoked function expression}
\newacronym{api}{API}{application programming interface}
\newacronym{gui}{GUI}{graphical user interface}
\newacronym{ux}{ux}{user experience}


\title{\pkg{rserve-ts}: a \pkg{TypeScript} interface to \pkg{Rserve} for modern web apps}
\author{Tom Elliott}
\affil{\small iNZight Analytics Ltd\\
Iwi affiliations: Ngāti Whakaue, Ngāti Raukawa}


\date{}

\begin{document}

\maketitle

\begin{abstract}
% TODO: Write 150-200 word abstract summarizing:
% - The problem of connecting modern web apps to R
% - The limitations of existing solutions
% - How rserve-ts solves these problems
% - Key benefits and improvements
\end{abstract}

\section{Introduction}
\label{sec:intro}

The \R{} programming language is a powerful tool for statistical computing and data analysis. As web applications become increasingly prevalent, there is a growing need to integrate \R{}'s capabilities into modern web applications. The \pkg{Rserve} package \parencite{Urbanek2003} provides a way for programs to utilize \R{}'s facilities through a client-server architecture. While various clients exist for different programming languages, the JavaScript ecosystem has relied on \pkg{rserve-js}, which, despite its stability, has not been significantly updated since 2018 and uses outdated JavaScript patterns.

This paper describes the development of \pkg{rserve-ts}, a modern TypeScript interface to \pkg{Rserve} designed for contemporary web applications. The library wraps \pkg{rserve-js} while providing a more modern interface that leverages promises for asynchronous code and implements comprehensive type-checking for data returned from \R{}. These improvements ensure runtime safety and enhance the development experience for front-end developers, making it easier to outsource front-end development to non-\R{} developers.

% TODO: Add brief paragraph about the broader context of R-to-web ecosystem improvements

\section{Background \& Motivation}
\label{sec:background}

\subsection{The R-to-Web Landscape}
\label{sec:landscape}

% TODO: Add overview of existing solutions for connecting R to web applications

\subsection{TypeScript in Modern Web Development}
\label{sec:typescript}

TypeScript has become increasingly important in modern web development, with usage growing from 12\% of developers in 2017 to 34\% in 2023 \parencite{JetBrains2023}. This growth reflects the language's ability to provide type safety for JavaScript applications, making development, extension, and debugging significantly easier. Type definitions not only catch potential runtime errors during development but also enable powerful development features like autocomplete.

\subsection{Limitations of Existing Solutions}
\label{sec:limitations}

The \pkg{rserve-js} package, while stable and actively used in projects like \proglang{RCloud}, employs outdated JavaScript coding patterns including:
\begin{itemize}
    \item Use of \glspl{iife}
    \item Lack of type safety
    \item Callback-based asynchronous code instead of modern promises
    \item Limited developer tooling support
\end{itemize}

These limitations make it increasingly difficult to integrate with modern web development workflows and tools.

\section{Core Design of rserve-ts}
\label{sec:design}

\subsection{Architecture Overview}
\label{sec:architecture}

The \pkg{rserve-ts} library operates in two modes:
\begin{enumerate}
    \item Standard evaluation: Users can send arbitrary \R{} code for evaluation
    \item Object capabilities (Ocaps): Results are automatically converted to JSON objects
\end{enumerate}

The library's \gls{api} is based on the JSON representation of \R{} objects, modifying the standard evaluation mode to consistently return JSON objects for a more uniform interface.

\subsection{Modernizing Asynchronous Operations}
\label{sec:async}

[Content from section 2.1 about callbacks vs promises]

\subsection{Type System Fundamentals}
\label{sec:type-system}

In \R{}, all objects are vectors of a certain type. The translation of these types to JavaScript/TypeScript presents several challenges:

\subsubsection{Scalar vs Vector handling}
\label{sec:scalar-vector}

\begin{itemize}
    \item \R{} vectors of length 1 without attributes are converted to scalar values
    \item All other cases return objects with additional properties (\cmd{r\_type} and optionally \cmd{r\_attributes})
\end{itemize}

\subsubsection{Schema Validation}
\label{sec:validation}

The library uses the \pkg{zod} library to define and validate object schemas, ensuring runtime type safety. For example:

\begin{lstlisting}
const intSchema = z.custom((data) => {
  if (data instanceof Int32Array) {
    return data.hasOwnProperty('r_type') &&
           data.r_type === 'int_array';
  }
  return false;
});
\end{lstlisting}

% TODO: Add more examples of schema validation and type checking

[Continue with remaining sections...]

\section{Type System Implementation}
\label{sec:implementation}

[Content from sections 4.1-4.3]

\section{Developer Experience}
\label{sec:dev-experience}

% TODO: Add sections on:
% - How type safety prevents runtime errors
% - Examples of caught type errors
% - Development time savings
% - IDE integration
% - Code completion features
% - Type hints
% - Documentation integration
% - Example workflows

\section{Future Work}
\label{sec:future}

% TODO: Add section on:
% - Planned R package for type generation
% - Additional type safety features
% - Performance optimizations
% - Community feedback and improvements

\section{Conclusion}
\label{sec:conclusion}

% TODO: Write conclusion summarizing:
% - Key improvements over rserve-js
% - Impact on R-to-web development
% - Future directions

\section*{Acknowledgements}

Supported by the Ngā Puanga Pūtaiao Fellowships from Government funding, administered by the Royal Society Te Apārangi.

\printglossaries
\printbibliography

\end{document}
