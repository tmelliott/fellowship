\documentclass{article}

% package for affiliation
\usepackage{authblk}
\usepackage{fullpage}
\usepackage{setspace}
\doublespacing

\newcommand{\pkg}[1]{\texttt{#1}}
\newcommand{\cmd}[1]{\texttt{#1}}
\newcommand{\prog}[1]{{\sf #1}}
\newcommand{\proglang}[1]{\prog{#1}}
\newcommand{\R}{\prog{R}}


\usepackage[hidelinks]{hyperref}

\title{iNZight Rebuild Workshop Report}
\author[1,2]{Elliott, T.M.\thanks{Ngāti Whakaue, Ngāti Raukawa}}
\author[1,2]{Sporle, A.A.\thanks{Ngāti Apa, Rangitāne, Te Rarawa}}
\author[1]{Wild, C.J.}
\affil[1]{iNZight Analytics Ltd}
\affil[2]{The University of Auckland}
\date{\small Workshops held 14 November and 5 December 2024\\
Report published TBA\\
Available from: URL
}

\usepackage[style=apa]{biblatex}
\addbibresource{../../../references.bib}

\begin{document}

\maketitle
\thispagestyle{empty}

\begin{abstract}
    The iNZight data analytics tool has been a popular choice for teaching statistics in New Zealand schools and universities. This report summarises the discussions and outcomes of two workshops held to discuss the rebuild of iNZight with statistics lecturers at The University of Auckland and Victoria University of Wellington. The aim of these workshops was to engage with stakeholders over the future direction of iNZight, with a focus on functionality that makes iNZight work for teaching statistics. While the primary focus was on the long term future, discussion inevitably touched on short-term issues that could be addressed in the current versions of iNZight to ensure it continues to provide for our users until such time as the rebuild is complete.
\end{abstract}

\vfill

\begin{center}
\thanks{\noindent Supported by the Ngā Puanga Pūtaiao Fellowships from
Government funding,\\administered by the Royal Society Te Apārangi.\\We would also like to thank the attendees of the workshops for their time and valuable input.}
\end{center}
\vspace{1em}


\section{Introduction}

iNZight has established itself as an essential tool for statistics education in New Zealand, offering a unique approach to data analysis and visualization \parencite{Wild2021,Wild2022}.
Originally developed at The University of Auckland, the software exists in two forms: a desktop version built using GTK \parencite{GTKTeam2020} and a web-based version using Shiny \parencite{Chang2021}.
However, the underlying technologies have become increasingly outdated, with GTK now deprecated and Shiny showing limitations in handling complex applications.
This technological obsolescence, combined with ongoing connectivity issues in the online version, has necessitated a complete rebuild of the platform.

This rebuild presents an opportunity to not only modernize iNZight's architecture but also to reimagine its capabilities while ensuring it continues to serve its core educational purpose.
To guide this process, we organized two workshops with statistics lecturers from The University of Auckland and Victoria University of Wellington.
These workshops aimed to address stakeholder concerns about iNZight's future and gather insights about which features are most valuable for statistics education, which could be improved, and what new functionality might enhance the teaching experience.


\section{Discussion summary}

The discussion was mostly unstructured, with attendees encouraged to speak freely and openly about their experiences with iNZight.
The following is a structured summary of some of the key points made during the workshops.

\subsection{What are your reasons for using iNZight?}

\begin{itemize}
    \item Change is hard: it is firmly built into some courses, and change would require a lot of work updating course notes, etc.
    \item The interface design supports learners through visualisation; visualisation first is fundamental to data analysis.
    \item Teaching versus learning versus research, although researchers still need to visualize data - other tools often have more research/industry focus and are less suitable for teaching.
    \item Immediacy of outputs: something happens immediately after you click a button, change an input, etc.
    \item The output provided, information displayed is more comprehensive that alternatives (e.g., Jamovi).
    \item Survey package support is easy to use and intuitive.
    \item Model fitting module for sample surveys.
    \item The fact that iNZight uses R makes it easy for students to transition from the UI to coding ---- same graphs and outputs to compare.
    \item Data mangement: rename variables interface is the best UI for doing this; could easily rename variables into your own language, etc.
    \item It doesn't try to do everything: just a few things really well.
    \item Import from a URL, this enables some cool teaching ideas.
    \item Ability to create graphics that users can recreate with code: any changes to this should be considered carefully, discussed idea of a "plot generator" that ensures interaction is always converted into function arguments for reproducibility.
\end{itemize}

\subsection{What's not so good?}

\begin{itemize}
    \item Inconsistencies, idiosyncracies: p-value, rounding not consistent throughout outputs; could be more/better consideration on our output is formatted, rounded.
    \item Some strange artefacts that mean outputs change over time, so automated teaching/assessment is difficult/needs constant updating.
    \item Code generation: not necessarily in a logical order (for teaching coding).
    \item No log of steps taken, reproducibility.
    \item Some sub-optimal procedures to do common things, when perhaps the default/standard isn't right/the best way to do it (possibly opinionated).
    \item The plot design for categorical versus numeric --- the order of variables \emph{does not} matter, but this does not necessarily translate to teaching modeeling concepts, etc.
    \item Better handling of errors when importing data, especially when weird characters are present. Robustness is required for teaching, as students import different/live datasets so some will inevitably run into issues.
    \item Plot types: ``spine plots'' not universally liked, understood, require more explanation --- option to turn off would be good.
\end{itemize}

\subsection{Features that would be nice}

\begin{itemize}
    \item Shareable link (e.g., Codap)
    \item Better separation of interface/global settings (e.g., font size) versus current chart (plot type, etc). \emph{Note: this may be more of a Lite issue}.
    \item Option to set/choose the first variable when sharing a link. \emph{Note: this is already a feature, but not documented well}.
    \item Reordering levels of a factor by drag-and-drop.
    \item Data cleaning using interactive graphics.
\end{itemize}

\subsection{Research opporunities}

\begin{itemize}
    \item How interface design aids learning.
    \item Year 12--13, reeducating teachers on using data and tools.
    \item From spreadsheet to iNZight to R.
\end{itemize}

\subsection{Barriers}

The main barrier for iNZight in future is competition from other tools, such as Codap and Jamovi.
Codap especially is for data science research and is heavily funded.


\subsection{Short-term issues}

As well as discussing the long-term future of iNZight, we also discussed some short-term issues that could be addressed in the current versions of iNZight to ensure it continues to provide for our users until such time as the rebuild is complete.

\begin{itemize}
    \item Two numeric variables: inference has no default (linear trend), user must specify it (unlike the other variable combinations) --- but things like this can be valuable teaching moments.
    \item Ability to set null hypothesis for linear regression.
    \item Confidence region for linear trend (not just bootstraps).
\end{itemize}

\subsubsection{iNZight Lite (shiny)}

\begin{itemize}
    \item Minor UI fixes to improve responsive design (e.g., for smaller screens).
    \item Known issues, e.g., trend option disappearing after saving residuals --- setting name of new variable doesn't seem relevant, additional clicking and produces the bug.
    \item Example data: autoload Visualize tab.
\end{itemize}


\section{Conclusion}

The workshops revealed that iNZight's strength lies in its focused approach to teaching statistics through visualization and its user-friendly interface.
Key advantages identified include its immediate visual feedback, comprehensive outputs, and seamless integration with R, which facilitates students' transition to coding.
While some technical improvements are needed, such as consistency in output formatting and better error handling, the core functionality remains highly valued by educators.

The feedback gathered suggests that future development should maintain iNZight's educational focus while addressing technical limitations.
Important considerations for the rebuild include improving reproducibility through better code generation and step logging, enhancing data cleaning capabilities, and implementing features for better collaboration such as shareable links.
However, it's crucial to maintain the software's current strengths: its immediate visual feedback, straightforward interface, and focused feature set that supports the teaching of statistical concepts.

Future workshops will be held with other local and international stakeholders to gather additional feedback and refine the rebuild plan.
These insights will guide the rebuild process, ensuring that iNZight continues to serve its educational mission while adapting to modern technical requirements and user needs.

\printbibliography

\end{document}
